\chapter{Background}

\section{Intro}
\paragraph{}We can think of two approaches to this problem: developmental approaches, in particular developmental robotics, and reinforcement learning (\rl). Developmental robotics takes inspirations from artificial intelligence, developmental psychology and neuroscience to model cognitive processes in natural and artificial systems \citep{asada2009cognitive,cangelosi2015developmental}. Following the idea that intelligence should be \textit{embodied}, robots are often used to test learning models. Reinforcement learning, on the other hand, is the field interested in problems where agents learn to behave by experiencing the consequences of their actions under the form of rewards and costs. As a result, these agents are not explicitly taught, they need to learn to maximize cumulative rewards over time by trial-and-error \citep{sutton2018reinforcement}. While developmental robotics is a field oriented towards answering particular questions around sensorimotor, cognitive and social development (e.g. how can we model language acquisition?), reinforcement learning is a field organized around a particular technical framework and set of methods.

\clearpage


\chapter{Introduction}

\minitoc
\todo{make glossary}


\paragraph{High Level motivations}

One fundamental goal of Artificial Intelligence (\ai) is to build generally capable agents~\citep{team2021open}.

To that end researchers in \ai takes several angles of attack and use different paradigms.

\begin{itemize}
\item Learning from observations (Supervised Learning)
\item Learning from exploration (Reinforcement Learning)
\item Learning from demonstration (Imitation Learning)
\item Learning from others (Multi-agent learning)
\end{itemize}
Learning from Social Interactions (Language and cultural ????). Perhaps the greatest source of information\,--\,the strongest signal for learning\,--\,is social interactions 

This manuscript is organized around two main questions:
%
\begin{enumerate}
	\item How can such a cultural model emerge in populations of agents?
	\item How can agents leverage existing cultural models to become better learners? 
\end{enumerate}

There is no open-endedness without culture. From an engineering perspective, building open-ended environments for artificial agents requires a community effort.

Culture as youtube videos (from perspective)

Citation (bruner, vygotsky)

\paragraph{A developmental Approach}

\paragraph{Objectives and Contributions}

\clearpage

\paragraph{Chronological order of publications}

\paragraph{Collaborations}



